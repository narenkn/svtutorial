\chapter{Literals}
Literals are fixed values in source code and normally does not
include spaces. \lsvlog language adds few literals including
time literal, array literal, structure literal, and
enhancements to string literal.

\section{Integer and Logic}
SV introduces unsized single bit state values \texttt{'0, '1, 'z, 'Z,
  'x, 'X}. Important feature to notice is that there is no format
specifier in the literal and it is devoid of width specification as
well. When assigned to a multibit data-type, they automatically size
to the vector's size. The feature aids portable code.
\begin{lstlisting}[label=lst!literal!logic,caption=Unsized literal usage]
logic [47:0] val48;

initial begin
  val48 = 'x;
  val48 = 48'hxxxxxxxxxxxx;
  val48 = 48'bxxxxxxxxxxxxxxxxxxxxxxxxxxxxxxxxxxxxxxxxxxxxxxxx;
end
\end{lstlisting}
Listing~\ref{lst!literal!logic} shows three different ways of
assigning a 48bit logic vector with unknown state value `x'. It is
very obvious that the unsized single bit format is more readable and
is not error prone.

\section{Time}
SV introduces accurate time specifications. Earlier method of
specification in \lvlog was prone to errors due to compiler
switches such as \kw{-override\_timescale} and the ordering of
\kw{`timescale} directives. Accurate time specification literal is a
real-value followed by time unit, without a space. Time unit is one of
\texttt{fs ps ns us ms s}. Examples of valid specifications are
\texttt{0.1ns, 1s, 3.141us}.

\section{Array}
Array literals are syntactically similar to C initializers, but with
the replicate operator ( {{}} ) allowed.  The nesting of braces must
follow the number of dimensions, unlike in C. However, replicate
operators can be nested.

\begin{table}
\begin{tabular}{p{0.6\textwidth} p{0.3\textwidth}}
\begin{lstlisting}[label=lst!literal!array, caption=Array Literal
    examples]
int n[1:2][1:3] = '{'{0,1,2},'{3{4}}};
int n[1:2][1:6] = '{2{'{3{4, 5}}}};
\end{lstlisting}
&
\begin{tbldesc}
\begin{verbatim}


Line 1: becomes '{'{0,1,2}, '{4, 4, 4}}
Line 2: becomes '{'{4,5,4,5,4,5},'{4,5,4,5,4,5}}
\end{verbatim}
\end{tbldesc}
\end{tabular}
\end{table}

Array literals can also use their index or type as a key and use a
default key value. Both lines of Listing~\ref{lst!literal!arr2} are
equivalent.
\begin{table}
\begin{tabular}{p{0.6\textwidth} p{0.3\textwidth}}
\begin{lstlisting}[label=lst!literal!arr2, caption=Array Literal with
    default value]
int b[0:3] = '{1:1, default:0};
int b[0:3] = '{1:1, int:0};
\end{lstlisting}
&
\begin{tbldesc}
\begin{verbatim}


Line 1, 2: both becomes [0, 1, 0, 0]
\end{verbatim}
\end{tbldesc}
\end{tabular}
\end{table}

\section{Structure}
A structure literal must have a type, which may be either explicitly
indicated with a prefix or implicitly indicated by an assignment-like
context. Nested braces should reflect the structure as in
Listing~\ref{lst!literal!struct}. Unordered values as in line 4, would
be an error.

\begin{table}
\begin{tabular}{p{0.6\textwidth} p{0.3\textwidth}}
\begin{lstlisting}[label=lst!literal!struct, caption=Structure Literals]
struct {int a; shortreal b;} ab, ab2[2];
ab = '{0, 0.0};
ab2 = '{'{1, 1.1}, '{2, 2.2}};
ab2 = '{1, 1.1, 2, 2.2};
ab = '{int:1, shortreal:1.1};
ab = '{default:0};

struct {int a, b[4];} ab [2][3] =
                '{2{'{3{a,'{2{b,c}}}}}};
\end{lstlisting}
&
\begin{tbldesc}
\begin{verbatim}


Line 4: Error: need to brace array elements
Line 5: becomes '{1, 1.1}
Line 6: becomes '{0, 0.0}
Line 8:
 '{2{ '{3{ a, '{b, c, b, c} }} }}
 '{2{ '{a, '{b,c,b,c}}, '{a, '{b,c,b,c}},
                      '{a, '{b,c,b,c}} }}
 '{ '{ '{a, '{b,c,b,c}}, '{a, '{b,c,b,c}},
                      '{a, '{b,c,b,c}} },
    '{ '{a, '{b,c,b,c}}, '{a, '{b,c,b,c}},
                      '{a, '{b,c,b,c}} } }
\end{verbatim}
\end{tbldesc}
\end{tabular}
\end{table}

\section{String}
\lsvlog adds few special characters to string literals such as
\texttt{\textbackslash{}a, \textbackslash{}v, \textbackslash{}f,
  \textbackslash{}x}. These special characters are intrepretted inside
a string literal enclosed within ``''. They have the same definition
as C quoted strings such as ``bell'', ``vertical feed'', ``form feed''
and ``hex number''. \texttt{\textbackslash{}x} should be followed by
hexadecimal value corresponding to the desired character.

\begin{table}
\begin{tabular}{p{0.6\textwidth} p{0.3\textwidth}}
\begin{lstlisting}[label=lst!literal!string, caption=String Literal
    Examples]
byte c3 [0:7] = "Hello World\n";
byte c4 [7:0] = "Hello World\n";
bit [10:0] a = "\x41";
bit [1:4][7:0] h = "hello";
\end{lstlisting}
&
\begin{tbldesc}
\begin{verbatim}


Line 1: c3[0] = "H", c3[7] = "o"
Line 2: c4[0] = "H", c4[7] = "H"
Line 3: assigns to a 'b000_0100_0001
Line 4: assigns to h "ello"
\end{verbatim}
\end{tbldesc}
\end{tabular}
\end{table}

A string literal can be assigned to an unpacked array of bytes. If the
size differs, it is left justified. In
Listing~\ref{lst!literal!string}, \texttt{c3[0]} contains character
`H' and \texttt{c3[7]} contains ASCII value of character `o'.
