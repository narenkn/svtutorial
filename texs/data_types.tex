\chapter{Data Types}
\lsvlog adds multiple integer data types, string, chandle, and class
data types and enhances the Verilog event type.

\section{Integer data types}
Integer data types could be either 2-state or 4-state types. Complete
list of integer data types are listed in Table~\ref{dt!table!idt}. All
integer data types are integral as they can be converted to or
assigned from a one-dimensional bit-vector. Other data types such as
\kwpacked \kw{array}, \kwpacked \kwstruct, \kwpacked \kwunion, \kwenum
and \kwtime do also exhibit the integral property.

\begin{table}[h]
\caption{Integer data types}
\centering
\begin{tabular}{|l|l|l|l|l|}
\hline
Data type & 2/4 state & Default Sign-ness & lvalue & rvalue \\
\hline \hline
\kwshortint & 2-state & signed & $-2^{15}$ & $2^{15}-1$ \\
\hline
\kwint & 2-state & signed & $-2^{31}$ & $2^{31}-1$ \\
\hline
\kwlongint & 2-state & signed & $-2^{63}$ & $2^{63}-1$ \\
\hline
\kwbyte & 2-state & signed & $-2^7$ & $2^7-1$ \\
\hline
\kwbit & 2-state & unsigned & \multicolumn{2}{l}{user-defined vector size} \\
\hline
\kwlogic & 4-state & unsigned & \multicolumn{2}{l}{user-defined vector size} \\
\hline
\kwreg & 4-state & unsigned & \multicolumn{2}{l}{user-defined vector size} \\
\hline
\kwinteger & 4-state & signed & $-2^{31}$ & $2^{31}-1$ \\
\hline
\kwtime & 4-state & unsigned & $0$ & $2^{64}-1$ \\
\hline
\end{tabular}
\label{dt!table!idt}
\end{table}

Every bit of a 2-state data-type could hold the value '0' or '1',
whereous each bit of a 4-state data-type could hold values '0', '1',
'X' or 'Z'.

All integer data types could be either \kwsigned or \kwunsigned. It becomes
important to understand sign-ness to understand the boundaries of each
data-type. When an integer data type is declared to be \kwunsigned, it
would have range from $0$ to $2^n-1$, where $n$ is the number of bits
in the type.

\section{Floating-point data types}
\lvlog already defines \kwreal, which is equivalent to \lc's
\kw{double} data-type. \lsvlog defines \kwshortreal data-type, that is
equivalent to \lc's \kw{float}. In general implementations of \lc
compilers, \kw{float} has 7 decimal precision, whereous \kw{double}
has a precision of 15.

\section{void}
The \kwvoid data type represents nonexistent data. This type can be
specified as the return type of functions to indicate no return
value. This type can also be used for members of tagged unions.

\section{chandle}
The \kwchandle data type represents storage for pointers. Its only
application is to store DPI returned pointers. \lsvlog can neither
create nor destroy \lc pointers. \kwchandle have initial value as
\kwnull. \kwchandle could be used in boolean equations to test for
equality \texttt{==, !=, ===, !==}. In boolean context, a \kwchandle
evaluates to 0 if \kwnull or to 1 otherwise. It should be obvious that
\kwchandle could be used as arguments to \kwfunction or \kwtask but
not as ports.

\section{string}
\lsvlog's \kwstring are dynamic as their length may vary during
simulation. They could be assigned from a string literal. \kwstring
data-type allows indexing single or a group of characters. Unlike \lc,
\lsvlog's \kwstring doesn't contain the special character
'\textbackslash{}0'.
